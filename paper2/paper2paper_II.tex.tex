%===========================%
%  Paper II (standalone)    %
%===========================%

% -- Self-contained bibliography (writes paper_II.bib on compile)
\begin{filecontents*}{paper_II.bib}
@article{AslanbeigiEtAl2014,
  author  = {Aslanbeigi, Siavash and Saravani, Mehdi and Sorkin, Rafael D.},
  title   = {Generalized causal set d’Alembertians},
  journal = {JHEP},
  year    = {2014},
  volume  = {2014},
  number  = {6},
  pages   = {24},
  doi     = {10.1007/JHEP06(2014)024}
}
@article{BenincasaDowker2010,
  author  = {Benincasa, Dionigi M. T. and Dowker, Fay},
  title   = {The Scalar Curvature of a Causal Set},
  journal = {Phys. Rev. Lett.},
  year    = {2010},
  volume  = {104},
  number  = {18},
  pages   = {181301},
  doi     = {10.1103/PhysRevLett.104.181301}
}
@inproceedings{Glaser2011DICE,
  author    = {Glaser, Lisa},
  title     = {The Spectral Dimension of Causal Sets},
  booktitle = {J. Phys.: Conf. Ser. (DICE 2010)},
  year      = {2011},
  volume    = {306},
  pages     = {012062},
  doi       = {10.1088/1742-6596/306/1/012062}
}
@phdthesis{Meyer1988,
  author  = {Meyer, David A.},
  title   = {The Dimension of Causal Sets},
  school  = {MIT},
  year    = {1988}
}
\end{filecontents*}

\documentclass[11pt]{article}

% --- packages
\usepackage[margin=1in]{geometry}
\usepackage{amsmath,amssymb,amsthm,mathtools}
\usepackage{graphicx}
\usepackage{hyperref}
\usepackage{microtype}
\usepackage[numbers,sort&compress]{natbib}
\usepackage{xcolor}
\usepackage{float}
\usepackage[section]{placeins}
\usepackage{bm}

% --- hyperlinks
\hypersetup{colorlinks=true, linkcolor=blue!50!black, citecolor=blue!50!black, urlcolor=blue!50!black}

% --- theorem envs & shorthands
\numberwithin{equation}{section}
\theoremstyle{plain}
\newtheorem{theorem}{Theorem}
\newtheorem{lemma}[theorem]{Lemma}
\theoremstyle{remark}
\newtheorem{remark}[theorem]{Remark}
\newcommand{\E}{\mathbb{E}}
\newcommand{\ellc}{\ell}

% --- figure paths + robust include
\graphicspath{{figs2/}{paper2/figs/}}
\newcommand{\SafeGraphic}[2]{%
  \IfFileExists{#1}{\includegraphics[width=#2\linewidth]{#1}}{%
    \fbox{\parbox[c][2.2in][c]{#2\linewidth}{\centering\small \textsf{Missing figure:}\\\texttt{#1}}}}}

\title{Paper II: Continuum Emergence and Invariance Tests\\
for Negentropic Birth--Death Causal Sets}
\author{Daniel J. Murray}
\date{\today}

\begin{document}
\maketitle

\begin{abstract}
We prove convergence of the BDG operator to the continuum d’Alembertian on 4D Minkowski for smooth probes, quantify discretization error rates, and test orientation invariance, spectral dimension, and Benincasa–Dowker (BD) scalar curvature on weakly curved controls. Each diagnostic has explicit acceptance thresholds; failure of any falsifies the model at this scope.
\end{abstract}

\paragraph{Run configuration.}
Seeds $=8$; size ladder $N\in\{2^{10},2^{11},2^{12},2^{13}\}$; medians and IQRs reported. Coarse length $\ellc\sim(\mathrm{Vol}/N)^{1/4}$.

\section{Convergence of $B_{\rm BDG}$ to $\Box$ on Minkowski}

Let $B_{\rm BDG}$ denote a 4D generalized causal-set d’Alembertian \citep{AslanbeigiEtAl2014}. For $f\in C_c^4(\mathbb{R}^{1,3})$, define the discrete evaluation $(B_{\rm BDG} f)(x)$ by sampling $f$ on sprinkled elements near $x$ and forming the BDG layer combination.

\begin{theorem}[BDG $\to \Box$ with an $L^2$ rate]\label{thm:bdg}
On 4D Minkowski Poisson sprinklings of density $\rho=\ellc^{-4}$ restricted to a finite diamond $D_R$, there exists $p>0$ and $C=C(f,R)$ such that
\[
\big\| (B_{\rm BDG} f) - \Box f \big\|_{L^2(D_R)} \le C\,\ellc^{\,p}
\]
with probability $\to 1$ as $\ellc\to 0$. Moreover, $p$ depends only on the BDG kernel moment cancellation order and $f$’s $C^4$ norm.
\end{theorem}

\noindent\emph{Proof sketch.}
(i) Use the BDG integral representation \citep{AslanbeigiEtAl2014} to write $\E[(B_{\rm BDG} f)(x)]$ as a finite sum of lightcone integrals with coefficients whose moment cancellations reproduce $\Box f(x)$; the deterministic bias is thus $O(\ellc^{q})$ for some $q>0$ depending on canceled moments. (ii) Use Poisson concentration for interval counts to bound fluctuations in each layer sum; a union bound over layers yields variance $O(\ellc^{4})$ times local $L^\infty$ norms of derivatives. (iii) Integrate the bias+variance bound over $D_R$ to obtain the $L^2$ rate. \qed

\begin{remark}
The same argument yields pointwise convergence away from the boundary with an $\ellc$-dependent boundary layer; our diagnostics always shave a collar from $D_R$ to avoid boundary artifacts.
\end{remark}

\section{Conformal bias on weakly curved controls}

Let $g=\Omega^2\eta$ with slowly varying $\Omega(\xi)=1+\epsilon\,\xi^2$.
\begin{lemma}[Conformal bias model]\label{lem:conf}
For $f\in C_c^4$ and small $\epsilon$, one has
\[
\Box_g f = \Omega^{-2}\Big(\Box f + (d-2)\,\Omega^{-1}\,\eta^{\mu\nu}\partial_\mu\Omega\,\partial_\nu f \Big) + O(\epsilon^2),
\]
so the BDG-vs-$\Box$ $L^2$ error on $g$ equals a linear bias in $\epsilon$ plus the flat-space discretization error $O(\ellc^{\,p})$.
\end{lemma}

\noindent\emph{Proof sketch.}
Linearize the conformal transformation of the Laplace–Beltrami operator and use smoothness of $f,\Omega$ to control remainders; BDG approximates $\Box$ to $O(\ellc^{\,p})$ by Theorem~\ref{thm:bdg}. \qed

\section{Diagnostics and falsifiers}

\subsection{Orientation invariance}
Random half-space partitions via scores $s(u)=a\,\deg_-(u)+b\,\deg_+(u)$ (i.i.d.\ $a,b$). Test statistic: coefficient of variation (CV) of cross-edges across partitions. \textbf{Acceptance:} CV is seed-stable and independent of orientation; non-increasing with refinement.

\begin{figure}[H]\centering
\SafeGraphic{lorentz_bootstrap.png}{0.80}
\caption{Orientation invariance CV (flat vs weakly curved controls). \textbf{Falsifier:} orientation-dependent CV beyond sampling error or upward trend with refinement.}
\end{figure}

\subsection{BDG vs continuum: $L^2$ rate}
Probe $f(\xi)=e^{-(t^2+|\mathbf{x}|^2)}$; report $E(\ellc)=\|B_{\rm BDG}f-\Box f\|_{L^2}$.
\begin{figure}[H]\centering
\SafeGraphic{bdg_error_curve.png}{0.80}
\caption{$L^2$ error vs $\ellc$ (log–log). \textbf{Acceptance:} slope $\hat p>0.5$ (95\% CI) and monotone decay. \textbf{Falsifier:} $\hat p\le 0$ or non-monotone.}
\end{figure}

\subsection{Spectral dimension (field proxy)}
Lazy random-walk return probability $P(\tau)$ on the undirected cover: $d_s(\tau)=-2\,\frac{d\log P}{d\log \tau}$.
\begin{figure}[H]\centering
\SafeGraphic{spectral_flow.png}{0.80}
\caption{Spectral dimension vs scale: UV reduction and mesoscopic plateau near 4 \citep{Glaser2011DICE,Meyer1988}. \textbf{Falsifier:} no plateau near $4$.}
\end{figure}

\subsection{BD scalar curvature (exact 4D layer form)}
Let $L_k(x)=|\{y\prec x:\ |I(y,x)|=k-1\}|$, $k=1,\dots,4$; $S^{(4)}(x)=1-L_1+9L_2-16L_3+8L_4$ \citep{BenincasaDowker2010}. \textbf{Acceptance:} on flat controls, median $|S^{(4)}|$ shrinks with $\ellc$; on weakly curved controls, the sign/trend matches the conformal bias model.
\begin{figure}[H]\centering
\SafeGraphic{bd_curvature_exact.png}{0.80}
\caption{Interval-abundance and $S^{(4)}$ summaries. \textbf{Falsifier:} wrong sign/trend or nonshrinking flat residuals.}
\end{figure}

\section*{Acknowledgments}
We thank the causal set community; code and data reside in \texttt{FUT\_toe-paper}.

\appendix

\section{Boundary collars and rate extraction}
We remove a boundary collar of thickness $\Theta(\ellc)$ before rate fitting; collars avoid mixing boundary bias into bulk rates.

% Ensure refs appear
\nocite{AslanbeigiEtAl2014,BenincasaDowker2010,Glaser2011DICE,Meyer1988}

% --- References last
\FloatBarrier
\clearpage
\bibliographystyle{unsrtnat}
\bibliography{paper_II}

\end{document}
