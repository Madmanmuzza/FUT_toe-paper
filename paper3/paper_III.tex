%===========================%
%  Paper III (standalone)   %
%===========================%

% -- Self-contained bibliography (writes paper_III.bib on compile)
\begin{filecontents*}{paper_III.bib}
@article{BenincasaDowker2010,
  author  = {Benincasa, Dionigi M. T. and Dowker, Fay},
  title   = {The Scalar Curvature of a Causal Set},
  journal = {Physical Review Letters},
  year    = {2010},
  volume  = {104},
  number  = {18},
  pages   = {181301},
  doi     = {10.1103/PhysRevLett.104.181301}
}

@article{AslanbeigiEtAl2014,
  author  = {Aslanbeigi, Siavash and Saravani, Mehdi and Sorkin, Rafael D.},
  title   = {Generalized causal set d'Alembertians},
  journal = {Journal of High Energy Physics},
  year    = {2014},
  volume  = {2014},
  number  = {6},
  pages   = {24},
  doi     = {10.1007/JHEP06(2014)024}
}

@inproceedings{Glaser2011DICE,
  author    = {Glaser, Lisa},
  title     = {The Spectral Dimension of Causal Sets},
  booktitle = {Journal of Physics: Conference Series (DICE 2010)},
  year      = {2011},
  volume    = {306},
  pages     = {012062},
  doi       = {10.1088/1742-6596/306/1/012062}
}

@article{Sorkin2007,
  author  = {Sorkin, Rafael D.},
  title   = {Does locality fail at intermediate length-scales?},
  journal = {arXiv:gr-qc/0703099},
  year    = {2007}
}

@phdthesis{Meyer1988,
  author  = {Meyer, David A.},
  title   = {The Dimension of Causal Sets},
  school  = {MIT},
  year    = {1988}
}
\end{filecontents*}

\documentclass[11pt]{article}

% --- packages
\usepackage[margin=1in]{geometry}
\usepackage{amsmath,amssymb,amsthm,mathtools}
\usepackage{graphicx}
\usepackage{hyperref}
\usepackage{microtype}
\usepackage[numbers,sort&compress]{natbib}
\usepackage{xcolor}
\usepackage{float}                % [H] hard placement
\usepackage[section]{placeins}    % Float barriers per section

% --- hyperlink setup
\hypersetup{
  colorlinks=true,
  linkcolor=blue!50!black,
  citecolor=blue!50!black,
  urlcolor=blue!50!black
}

% --- numbering & theorem envs
\numberwithin{equation}{section}
\theoremstyle{plain}
\newtheorem{theorem}{Theorem}
\newtheorem{proposition}[theorem]{Proposition}
\theoremstyle{definition}
\newtheorem{definition}[theorem]{Definition}
\theoremstyle{remark}
\newtheorem{remark}[theorem]{Remark}

% --- shorthands
\newcommand{\E}{\mathbb{E}}
\newcommand{\Var}{\mathrm{Var}}
\newcommand{\Boxop}{\Box}
\newcommand{\MM}{d_{\mathrm{MM}}}

% --- figure paths
\graphicspath{{figs3/}{paper3/figs/}}

% --- robust include (single check)
\newcommand{\SafeGraphic}[2]{% #1 file, #2 width frac
  \IfFileExists{#1}{\includegraphics[width=#2\linewidth]{#1}}{%
    \fbox{\parbox[c][2.2in][c]{#2\linewidth}{\centering\small \textsf{Missing figure:}\\\texttt{#1}}}%
  }%
}

\title{Paper III: Effective Action and Einstein Limit\\
for Negentropic Birth--Death Causal Sets}
\author{Daniel J. Murray}
\date{\today}

\begin{document}
\maketitle

\begin{abstract}
We define a discrete gravitational action on the negentropic birth--death (BD) causal sets of Paper~I and show it reduces to the Einstein--Hilbert (EH) action with a cosmological term in the continuum limit. Using the operator/curvature diagnostics from Paper~II, we match constants to identify $(G,\Lambda)$ and then test three regimes: FLRW (Friedmann equations), Newtonian (Poisson limit), and linearized waves (luminal dispersion). Any failure falsifies the model at the classical level.
\end{abstract}

\section{Aim and falsification}
\textbf{Aim.} Exhibit a discrete action $S_{\mathrm{BD}}$ on BD causal sets that yields EH in the continuum and passes FLRW/Newtonian/linear tests with a single parameter identification $(\alpha,\beta)\mapsto(G,\Lambda)$.

\medskip\noindent\textbf{Falsification (any one fails):}
\begin{itemize}
\item Continuum reduction does not match $\frac{1}{16\pi G}\!\int R\sqrt{-g}+\Lambda\!\int\!\sqrt{-g}$ with constants consistent across controls;
\item Friedmann relation fails with the same $(G,\Lambda)$;
\item Newtonian limit does not reproduce Poisson;
\item Linearized dispersion deviates from $\omega^2=k^2$ (extra pole or mass term).
\end{itemize}

\section{Discrete action}
Let $(\mathcal{C},\prec)$ be a BD causal set. For $x\in\mathcal{C}$ define the 4D past-layer counts
\begin{equation}
L_k(x)=\big|\{\,y\prec x:\,|I(y,x)|=k-1\,\}\big|\quad (k=1,\dots,4),
\end{equation}
and the Benincasa--Dowker 4D curvature density \citep{BenincasaDowker2010}
\begin{equation}
\label{eq:BDdensity}
\mathcal{S}^{(4)}(x)\;=\;1 - L_1(x) + 9\,L_2(x) - 16\,L_3(x) + 8\,L_4(x).
\end{equation}
Define the action
\begin{equation}
\label{eq:SBD}
S_{\mathrm{BD}}(\mathcal{C}) \;=\; \alpha \sum_{x\in\mathcal{C}} \mathcal{S}^{(4)}(x) \;+\; \beta\,N \;+\; \gamma\,\mathrm{bdry}(\mathcal{C}),
\end{equation}
where $N=|\mathcal{C}|$, $\alpha,\beta,\gamma$ are constants, and $\mathrm{bdry}$ collects boundary terms (vanishing for compact regions or controlled slabs). The coarse length is $\ell \sim (\mathrm{Vol}/N)^{1/4}$.

\section{Continuum reduction and matching}
On conformally flat backgrounds $g=\Omega(\xi)^2\eta$ with $\Omega(\xi)=1+\epsilon\,\xi^2$ (small curvature), Paper~II established that
\begin{equation}
\label{eq:cont-limit}
\E\big[\mathcal{S}^{(4)}(x)\big] \;=\; c_R\,R(x)\,\ell^0 \;+\; O(\ell^p) \qquad (p>0),
\end{equation}
for a dimensionless constant $c_R$ fixed by the BD prescription and layer definition.\footnote{The $O(\ell^p)$ accounts for discretization and finite-size bias; Paper~II measured $p>0$ empirically.}
Averaging and summing gives
\begin{equation}
\label{eq:EHmatch}
\E\big[S_{\mathrm{BD}}\big] \;=\; \alpha\,c_R \!\int R\,\sqrt{-g}\,d^4x \;+\; \beta \!\int \sqrt{-g}\,d^4x
\;+\; O(\ell^p) \;+\; \gamma\,\E[\mathrm{bdry}],
\end{equation}
hence the identification
\begin{equation}
\label{eq:constants}
\frac{1}{16\pi G}=\alpha\,c_R,\qquad \Lambda=\beta.
\end{equation}
Constants $(\alpha,\beta)$ are fitted once using the Paper~II curved controls (Appendix~\ref{app:matching}).

\section{Coarse variation and effective equations}
Let $\rho$ denote a local order-density surrogate for $\sqrt{-g}$ inferred from interval counts. Under admissible local deformations that preserve causal order type, the first variation of \eqref{eq:SBD} yields, after coarse-graining and using \eqref{eq:constants},
\begin{equation}
\label{eq:EFE}
G_{\mu\nu}+\Lambda g_{\mu\nu} \;=\; 8\pi G\,T^{\mathrm{eff}}_{\mu\nu} \;+\; O(\ell^p),
\end{equation}
where $T^{\mathrm{eff}}$ collects BD-induced matter surrogates (birth--death asymmetries, conserved at the coarse level). The $O(\ell^p)$ remainder vanishes in the continuum limit.

\section{Empirical tests (falsifiable)}
\subsection*{FLRW (Friedmann)}
On $g=\Omega(t)^2\eta$ we estimate $H(t)$ from time-slab interval scalings and $\rho(t)$ from order-density. With $(G,\Lambda)$ fixed by matching, we test
\begin{equation}
H^2(t)\;=\;\frac{8\pi G}{3}\,\rho(t)\;+\;\frac{\Lambda}{3}\;+\;O(\ell^p).
\end{equation}
\begin{figure}[H]\centering
\SafeGraphic{friedmann_check.png}{0.80}
\caption{Friedmann check: $H^2$ vs.\ $\rho$ with fitted $(G,\Lambda)$ line.}
\end{figure}

\subsection*{Newtonian limit (Poisson)}
For a weak, static overdensity we estimate $\Phi$ from random-walk return bias and check
\begin{equation}
\nabla^2\Phi \;=\; 4\pi G\,\rho \;+\; O(\ell^p).
\end{equation}
\begin{figure}[H]\centering
\SafeGraphic{newtonian_residuals.png}{0.80}
\caption{Poisson residuals $\nabla^2\Phi-4\pi G\rho$ across radii; $O(\ell^p)$ envelope.}
\end{figure}

\subsection*{Linearized waves}
Driving a small plane-wave perturbation of link/interval statistics, the BDG response yields $\omega(k)$; we test
\begin{equation}
\omega^2 \;=\; k^2 \;+\; O(\ell^p).
\end{equation}
\begin{figure}[H]\centering
\SafeGraphic{linear_wave_dispersion.png}{0.80}
\caption{Dispersion $\omega^2$ vs.\ $k^2$; target $\omega^2=k^2$ with seed variability.}
\end{figure}

% --- Flush any remaining floats before appendices
\FloatBarrier
\clearpage

\section{Discussion}
The discrete action \eqref{eq:SBD} furnishes a parameter-minimal classical dynamics consistent with EH to leading order. The FLRW, Newtonian, and linearized tests probe distinct combinations of $(G,\Lambda)$ and discretization errors; consistent passage across scales supports the continuum reduction. Any systematic failure isolates whether the action kernel or BD normalization must be revised.

\section*{Acknowledgments}
We thank the causal set community; code and data reside in \texttt{FUT\_toe-paper}.

\appendix

\section{Units and scaling}
Let $\ell\sim(\mathrm{Vol}/N)^{1/4}$ be the coarse length. Then $\sum_x \mathcal{S}^{(4)}(x)$ is dimensionless in the BD normalization, while $\int R\sqrt{-g}\,d^4x$ has dimensions of length$^{\,2}$. Hence $\alpha$ carries dimensions of $(16\pi G)^{-1}/c_R$ with $c_R$ dimensionless, and $\beta$ shares the dimensions of $\Lambda$.

\section{Matching constants}\label{app:matching}
On conformal controls used in Paper~II, we measure $\E[\mathcal{S}^{(4)}]$ against the continuum $R$ and fit $c_R$. With \eqref{eq:EHmatch} this gives $(\alpha,\beta)$ via \eqref{eq:constants}. The boundary term is verified negligible by enlarging slabs until convergence.

\section{Error bounds and finite-size effects}
Empirically Paper~II found $E(\ell)=O(\ell^p)$ for the BDG operator error and a linear curvature bias in $\epsilon$; these transfer to the action density by linearity. For global actions, variance decays as $O(N^{-1/2})$ across seeds assuming weak dependence, improving the rate effectively to $O(\ell^{p})$ in mean with shrinking confidence bands.

% --- Ensure standard references appear even if not cited explicitly
\nocite{AslanbeigiEtAl2014,Glaser2011DICE,Sorkin2007,Meyer1988}

% --- References last
\FloatBarrier
\clearpage
\bibliographystyle{unsrtnat}
\bibliography{paper_III}

\end{document}
