%===========================%
%  Paper I (standalone)     %
%===========================%

% -- Self-contained bibliography (writes paper_I.bib on compile)
\begin{filecontents*}{paper_I.bib}
@article{Bombelli1987,
  author  = {Bombelli, L. and Lee, J. and Meyer, D. and Sorkin, R.},
  title   = {Space-time as a causal set},
  journal = {Phys. Rev. Lett.},
  year    = {1987},
  volume  = {59},
  pages   = {521--524},
  doi     = {10.1103/PhysRevLett.59.521}
}
@phdthesis{Meyer1988Dim,
  author  = {Meyer, David A.},
  title   = {The Dimension of Causal Sets},
  school  = {MIT},
  year    = {1988}
}
@article{Myrheim1978,
  author  = {Myrheim, Jan},
  title   = {Statistical Geometry},
  journal = {CERN preprint TH-2538},
  year    = {1978}
}
@article{BenincasaDowker2010,
  author  = {Benincasa, Dionigi M. T. and Dowker, Fay},
  title   = {The Scalar Curvature of a Causal Set},
  journal = {Phys. Rev. Lett.},
  year    = {2010},
  volume  = {104},
  number  = {18},
  pages   = {181301},
  doi     = {10.1103/PhysRevLett.104.181301}
}
@article{AslanbeigiEtAl2014,
  author  = {Aslanbeigi, Siavash and Saravani, Mehdi and Sorkin, Rafael D.},
  title   = {Generalized causal set d'Alembertians},
  journal = {JHEP},
  year    = {2014},
  volume  = {2014},
  number  = {6},
  pages   = {24},
  doi     = {10.1007/JHEP06(2014)024}
}
@book{LevinPeresWilmer2009,
  author  = {Levin, David A. and Peres, Yuval and Wilmer, Elizabeth L.},
  title   = {Markov Chains and Mixing Times},
  publisher = {American Mathematical Society},
  year    = {2009}
}
@book{MeynTweedie2009,
  author  = {Meyn, Sean P. and Tweedie, Richard L.},
  title   = {Markov Chains and Stochastic Stability},
  edition = {2nd},
  publisher = {Cambridge University Press},
  year    = {2009}
}
@article{Doeblin1937,
  author  = {Doeblin, Wolfgang},
  title   = {Remarques sur la th\'eorie m\'etrique des produits de cha\^ines de Markoff},
  journal = {Bull. Soc. Math. France},
  year    = {1937},
  volume  = {65},
  pages   = {132--148}
}
\end{filecontents*}

\documentclass[11pt]{article}

% --- packages
\usepackage[margin=1in]{geometry}
\usepackage{amsmath,amssymb,amsthm,mathtools}
\usepackage{graphicx}
\usepackage{hyperref}
\usepackage{microtype}
\usepackage[numbers,sort&compress]{natbib}
\usepackage{xcolor}
\usepackage{float}
\usepackage[section]{placeins}
\usepackage{bm}

% --- hyperlinks
\hypersetup{
  colorlinks=true,
  linkcolor=blue!50!black,
  citecolor=blue!50!black,
  urlcolor=blue!50!black
}

% --- theorem envs & shorthands
\numberwithin{equation}{section}
\theoremstyle{plain}
\newtheorem{theorem}{Theorem}
\newtheorem{lemma}[theorem]{Lemma}
\newtheorem{proposition}[theorem]{Proposition}
\theoremstyle{definition}
\newtheorem{definition}[theorem]{Definition}
\theoremstyle{remark}
\newtheorem{remark}[theorem]{Remark}

\newcommand{\E}{\mathbb{E}}
\newcommand{\ellc}{\ell} % coarse length

% --- figures
\graphicspath{{figs1/}{paper1/figs/}}
\newcommand{\SafeGraphic}[2]{%
  \IfFileExists{#1}{\includegraphics[width=#2\linewidth]{#1}}{%
    \fbox{\parbox[c][2.0in][c]{#2\linewidth}{\centering\small \textsf{Missing figure:}\\\texttt{#1}}}}}

\title{Paper I: Negentropic Birth--Death Dynamics and Foundational Diagnostics\\
for Causal Sets (with a Doeblin/ Coupling Proof of Well-posedness)}
\author{Daniel J. Murray}
\date{\today}

\begin{document}
\maketitle

\begin{abstract}
We define a normalized, saturated negentropic birth--death (BD) dynamics for causal sets on a finite Alexandrov interval and give a full Doeblin/coupling proof of well-posedness: the labeled Hasse-diagram chain is uniformly ergodic with a unique stationary distribution, hence the unlabeled chain has a unique invariant measure. We then calibrate core observables (interval abundances, Myrheim--Meyer dimension, link/degree statistics) against 4D Minkowski sprinklings and state falsifiers.
\end{abstract}

\paragraph{Run configuration.}
Seeds $=8$; sizes $N\in\{2^{10},2^{11},2^{12},2^{13}\}$. Coarse length $\ellc\sim(\mathrm{Vol}/N)^{1/4}$. Report medians and IQRs.

%==========================================================
\section{Dynamics and assumptions}
Let $D\subset \mathbb{M}^{1,3}$ be a fixed finite Alexandrov interval. A \emph{state} is a labeled Hasse diagram $(V,\prec)$ embedded in $D$ with $|V|\le N_{\max}$ (defined below). One BD step is either a \emph{birth} (insert a new element $x\in D$ and draw links consistent with $\prec$ by a local rule) or a \emph{death} (delete a locally saturated redundancy), with probabilities depending on interval statistics in a bounded neighborhood.

We impose minimal nondegeneracy/regularity assumptions on the \emph{normalized, saturated} BD kernel $K$:
\begin{itemize}
\item[(A1) Locality.] Birth/death decisions and link draws depend only on order neighborhoods within a fixed radius (layer depth) $r$.
\item[(A2) Positivity bounds.] There exist constants
\[
\underline{p}_{\rm add},\ \underline{p}_{\rm del},\ \underline{p}_{\rm link},\ \underline{p}_{\rm loc}\in(0,1)
\]
such that whenever a birth (or deletion, respectively) is admissible, it is chosen with probability at least the corresponding lower bound; conditional link decisions have probability at least $\underline{p}_{\rm link}$ for each admissible parent/child choice; the location proposal for a new element has density lower bounded by $\underline{p}_{\rm loc}/\mathrm{Vol}(D)$ on $D$.
\item[(A3) Saturation (finite local density).] There exists $N_{\max}<\infty$ such that if $|V|\ge N_{\max}$, then a deletion is admissible and is chosen with probability at least $\underline{p}_{\rm del}$ until $|V|<N_{\max}$. (Equivalently: the kernel enforces a hard or soft cap that prevents unbounded growth.)
\item[(A4) Aperiodicity (lazy variant).] For theoretical guarantees we analyze the $1/2$-lazy kernel $K_\ell=\tfrac12(I+K)$, which shares the same invariant measure and improves to aperiodicity \cite{LevinPeresWilmer2009}.
\end{itemize}
Assumptions (A1)--(A3) are satisfied by the negentropic BD rules used throughout (local scoring of interval patterns; normalized choices; saturation pruning). We will prove uniform ergodicity for the labeled chain; invariance under relabeling then induces a unique measure on unlabeled causal sets.

%==========================================================
\section{Well-posedness via Doeblin minorization and coupling}
Let $\mathsf{S}$ be the finite state space of labeled Hasse diagrams $(V,\prec)$ with $|V|\le N_{\max}$ in $D$. All proofs below refer to the lazy kernel $K_\ell$ (henceforth $K$).

\begin{lemma}[Finite state]\label{lem:finite}
Under (A3) the number of elements is bounded by $N_{\max}$; for each $n\le N_{\max}$ the number of labeled partial orders on $n$ elements is finite. Hence $|\mathsf{S}|<\infty$.
\end{lemma}

\begin{lemma}[Uniform hit to an atom]\label{lem:toatom}
Fix the \emph{atom} state $o$ (the one-element poset). Under (A2)--(A3), there exists $t_1\le N_{\max}-1$ and $\varepsilon_1>0$ such that for all $x\in\mathsf{S}$,
\[
K^{t_1}(x,\{o\}) \ \ge\ \varepsilon_1 \ :=\ (\tfrac12\,\underline{p}_{\rm del})^{\,N_{\max}-1}.
\]
\emph{Proof.} From any $x$ with $n$ elements, successive deletions reduce $n\to n-1$ until $1$. Each deletion happens with probability at least $\tfrac12\underline{p}_{\rm del}$ under laziness; at most $N_{\max}-1$ steps are needed. Multiply the lower bounds. \qed
\end{lemma}

\begin{lemma}[Uniform regeneration from the atom]\label{lem:fromatom}
There exists $t_2\le N_{\max}-1$ and $\varepsilon_2>0$ such that for all measurable $A\subseteq\mathsf{S}$,
\[
K^{t_2}(o,A)\ \ge\ \varepsilon_2\,\nu(A), \qquad \nu:=K^{t_2}(o,\cdot),
\]
with $\varepsilon_2 := \big(\tfrac12\,\underline{p}_{\rm add}\,\underline{p}_{\rm loc}\,\underline{p}_{\rm link}^{\,c_r}\big)^{N_{\max}-1}$, where $c_r$ bounds the number of link decisions per birth under (A1).
\emph{Proof.} Fix any target state $y$ with $m\le N_{\max}$ elements and a canonical order of insertions respecting $\prec_y$. Starting at $o$, in each step: choose birth (prob.\ $\ge\tfrac12\underline{p}_{\rm add}$), sample location (density $\ge \underline{p}_{\rm loc}/\mathrm{Vol}(D)$ in a pre-chosen ball), and realize the required local links (each with prob.\ $\ge\underline{p}_{\rm link}$). The product lower bound gives a uniform minimal probability for the canonical path. Taking $t_2=m-1\le N_{\max}-1$ and defining $\nu$ as stated yields the minorization from $o$. \qed
\end{lemma}

\begin{theorem}[Doeblin minorization, uniform ergodicity, uniqueness]\label{thm:doeblin}
Under (A1)--(A4), there exist $T=t_1+t_2$ and $\varepsilon=\varepsilon_1\varepsilon_2>0$ such that for all $x\in\mathsf{S}$ and measurable $A\subseteq \mathsf{S}$,
\[
K^{T}(x,A)\ \ge\ \varepsilon\,\nu(A),
\]
with $\nu$ as in Lemma~\ref{lem:fromatom}. Consequently, $K$ is uniformly ergodic with a unique stationary distribution $\pi$; for all $x$,
\[
\|K^{n}(x,\cdot)-\pi\|_{\mathrm{TV}} \ \le\ (1-\varepsilon)^{\lfloor n/T\rfloor}.
\]
\emph{Proof.} Compose Lemmas~\ref{lem:toatom} and \ref{lem:fromatom}: $K^{t_1}(x,\{o\})\ge \varepsilon_1$ and $K^{t_2}(o,\cdot)\ge \varepsilon_2\,\nu(\cdot)$ imply $K^{T}(x,\cdot)\ge \varepsilon\,\nu(\cdot)$ with $T=t_1+t_2$. This is a global Doeblin condition \cite{Doeblin1937,LevinPeresWilmer2009}, implying uniform (geometric) ergodicity and uniqueness of the stationary distribution. \qed
\end{theorem}

\begin{remark}[Coupling interpretation and mixing time]
Construct a synchronous coupling that forces both chains to hit $o$ within $t_1$ with probability $\ge\varepsilon_1$, then regenerate jointly for $t_2$ using the same randomness from $\nu$. Each epoch of length $T$ coalesces with probability $\varepsilon$; hence
$\tau_{\mathrm{mix}}(\delta)\ \le\ T\,\lceil \log(\delta^{-1})/\log((1-\varepsilon)^{-1})\rceil$ \cite{LevinPeresWilmer2009}.
\end{remark}

\begin{proposition}[Unlabeled uniqueness]
Because $K$ is invariant under relabelings, the unique labeled stationary distribution $\pi$ projects to a unique invariant measure on unlabeled causal sets (isomorphism classes).
\end{proposition}

\paragraph{Summary.} On any fixed finite region $D$, the normalized+saturated BD dynamics yields a finite, aperiodic Markov chain satisfying a global Doeblin condition. The chain is uniformly ergodic with unique $\pi$; all time averages converge almost surely (Birkhoff) and seed-averaged diagnostics are well-defined.

%==========================================================
\section{Foundational diagnostics and acceptance criteria}

\subsection{Interval abundance spectrum}
Let $n_k(x)$ be the count of subintervals of size $k$ ending at $x$; define $L_k=\E[\tfrac{1}{|V|}\sum_x n_k(x)]$, $k=1,\dots,4$. In 4D Minkowski Poisson sprinklings the combination
\[
S^{(4)}=1-L_1+9L_2-16L_3+8L_4
\]
vanishes on flat space \citep{BenincasaDowker2010}. \textbf{Acceptance:} on flat controls, median $|S^{(4)}|$ decreases with refinement; log–log slope $p>0.5$ (95\% CI).

\begin{figure}[H]\centering
\SafeGraphic{interval_hist.png}{0.80}
\caption{Interval-size histogram (flat control). \textbf{Falsifier:} persistent bias in $S^{(4)}$ or nonshrinking residuals.}
\end{figure}

\subsection{Myrheim--Meyer dimension}
Estimate $d_{\rm MM}$ from order-fraction statistics \citep{Myrheim1978,Meyer1988Dim}.
\textbf{Acceptance:} median $d_{\rm MM}=4\pm0.1$ at top size; error decreases with $N$.

\begin{figure}[H]\centering
\SafeGraphic{mm_dimension.png}{0.80}
\caption{Myrheim--Meyer dimension vs $N$. \textbf{Falsifier:} drift away from $4$ or nonshrinking error bars.}
\end{figure}

\subsection{Link/degree statistics and Lorentz symmetry}
Degree distributions and link densities should match Poisson sprinklings in 4D \citep{Bombelli1987}.
\textbf{Acceptance:} Kolmogorov–Smirnov distance to sprinkling baseline decreases with refinement.

\begin{figure}[H]\centering
\SafeGraphic{bd_rate_stability.png}{0.80}
\caption{Rate stability: link density and in/out-degree summaries across the ladder. \textbf{Falsifier:} nonstationary drift or seed-instability not shrinking with $N$.}
\end{figure}

%==========================================================
\section{Operators (preview)}
We record the generalized d'Alembertian $B_{\rm BDG}$ structure for use in Papers II–IV \citep{AslanbeigiEtAl2014}; calibration to flat controls is performed in Paper II.

\section*{Acknowledgments}
We thank the causal set community; code and data reside in \texttt{FUT\_toe-paper}.

\appendix

\section{Why laziness is harmless}
Replacing $K$ by $K_\ell=\tfrac12(I+K)$ preserves invariant measures and improves to aperiodicity; the Doeblin constant only improves (or equals) under laziness \cite{LevinPeresWilmer2009,MeynTweedie2009}.

\section{Constants and explicit bounds}
With $T=(N_{\max}-1)+(N_{\max}-1)$,
\[
\varepsilon = \big(\tfrac12\,\underline{p}_{\rm del}\big)^{N_{\max}-1}\,
             \big(\tfrac12\,\underline{p}_{\rm add}\,\underline{p}_{\rm loc}\,\underline{p}_{\rm link}^{\,c_r}\big)^{N_{\max}-1}.
\]
All are strictly positive by (A2)--(A3); hence the geometric rate $(1-\varepsilon)^{\lfloor n/T\rfloor}$ is explicit.

% Ensure references appear
\nocite{Bombelli1987,Myrheim1978,Meyer1988Dim,BenincasaDowker2010,AslanbeigiEtAl2014,LevinPeresWilmer2009,MeynTweedie2009,Doeblin1937}

% --- References last
\FloatBarrier
\clearpage
\bibliographystyle{unsrtnat}
\bibliography{paper_I}

\end{document}
